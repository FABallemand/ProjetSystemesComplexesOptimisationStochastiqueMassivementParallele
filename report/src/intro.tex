\section{Introduction}
Evolutionary algorithms and genetic programming are used in a wide range of
applications. From basic applications in industry and engineering to complex
problem resolution in resarch department, these two methods, whose development
started with Alan Turing in 1950, no longer need to prove their efficiency.\\

% State of the art?

On the one hand, evolutionary algorithms, inspired by biological evolution,
allows us to find quatities or caracteristics of an object, and on the other
hand, genetic programming was designed to find interactions between objects. By
combining these two methods, it is theoredicaly possible (assuming there is a
good way to evaluate our results) to find not only caracteritics defining two
entities, but also the way they interact with each other.
These two entities could very well be daily life objects or even animals. We
could actually be able to find how birds interact with each other in a flock.
These two entites could be less concrete and be ridiculously small: let us find
out how electrons interact in matter! But they could also be enormous like
celestial bodies...\\

Let us go back in time and imagine we only have the knowledge and tools to
measure the position of the Sun relatively to the Earth. Would we be able to
find caracteritics about the Earth or the Sun like their mass or their speed?
Is there a way to find out how they interact?\\

The goal of this project is to utilise the two afore mentionned programming
methods to retrieve caracteristics and laws regarding the interactions between
two celestial bodies.\\
We will focus on Earth and Sun as there is a strong interaction between them
and yet are comletely different.\\
First, we will conduct a physics study regarding the celestial bodies in order
to evaluate results to come.
In a second time, we will attempt to find caracteristics about the two planets using
evolutionary algorithms. Finally, we will use genetic programming to
re-discover Newton's law
of universal gravitation.

% Try with another planet as an exemple of generality~
% Use multi-thread/core and compare perfomrance ?