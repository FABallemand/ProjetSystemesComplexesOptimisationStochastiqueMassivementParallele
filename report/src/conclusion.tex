\section{Conclusion}

In conclusion, we first used evolutionary algorithms to find the mass of the
Earth, the mass of the Sun and the speed at which the Earth revolves around the
Sun. Evolutionnary Algotihms proved to be able to solve these problems without
any issue and in a short time as long as there is enough genetic diversity in
the dataset. This diversity comes from a large initial population and a
set of parameters defining the genetic operators.\\

The second part of the project was dedicated to the recovery of Newton's
universal law of gravitation. However, during this part, EASEA prevented us
from implementing the solution we found the most relevant. We then had to
backtrack and use a
different approach to solve the problem. This approach was successful, but it
was not as effective as we had originally planned. Nevertheless, we were able
to complete the project and obtain valuable information about the process, such
as the influence of the parameter on the convergence speed and the quality of
the final results.\\

Now that we know that genetic programming is capable of finding a physical law
from simple observations and constants, it would be interesting to see if it is
also capable of finding more complex models such as the Navier-Stokes equations,
which describe fluid flow, perhaps by taking advantage of the acceleration
offered by island parallelism and GPGPUs.\\